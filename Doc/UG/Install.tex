\section{Installation}

A manually written {\tt Makefile} is provided. At the topmost part of the
{\tt Makefile} you need to specify which {\tt make.inc} file you
want to use. You need to decide which compiler to use if there
are more than one compiler in you system.
In the directory {\tt platform} there are several {\tt make.inc} files.
Currently, {\tt ffr-LFDFT} is tested using the following compilers
on Linux system:
\begin{itemize}
\item GNU Fortran compiler
\item G95 Fortran compiler
\item Intel Fortran compiler
\item PGI Fortran compiler
\item Sun (now part of Oracle) Fortran compiler
\end{itemize}
For typical Linux system, {\tt make.inc.gfortran} is sufficient.
You can manually edit the compiler options in the corresponding {\tt make.inc}
files.

There following external libraries are required to build \ffrLFDFT
\begin{itemize}
\item BLAS
\item LAPACK
\item FFTW3
\end{itemize}

Typing the command
\begin{minted}{text}
make
\end{minted}
will build the library {\tt libmain.a} and typing
the command
\begin{minted}{text}
make main
\end{minted}
will build the main executable {\tt ffr\_LFDFT.x}.


