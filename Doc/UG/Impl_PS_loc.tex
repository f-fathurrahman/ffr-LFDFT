\subsection{Local pseudopotential}

Initialization of local potential for periodic system.
\begin{fortrancode}
ALLOCATE( ctmp(Npoints) )
V_ps_loc(:) = 0.d0
DO isp = 1,Nspecies
  ctmp(:) = cmplx(0.d0,0.d0,kind=8)
  DO ip = 1,Npoints
    Gm = sqrt(G2(ip))
    ctmp(ip) = hgh_eval_Vloc_G( Ps(isp), Gm ) * strf(ip,isp) / Omega
  ENDDO
  ! inverse FFT: G -> R
  CALL fft_fftw3( ctmp, Nx, Ny, Nz, .true. )
  ! XXX: Move this outside isp loop ?
  DO ip = 1,Npoints
    V_ps_loc(ip) = V_ps_loc(ip) + real( ctmp(ip), kind=8 )
  ENDDO 
ENDDO 
\end{fortrancode}

Subroutine to calculate structure factor:
\begin{fortrancode}
shiftx = 0.5d0*( grid_x(2) - grid_x(1) )
shifty = 0.5d0*( grid_y(2) - grid_y(1) )
shiftz = 0.5d0*( grid_z(2) - grid_z(1) )
ALLOCATE( strf(Ng,Nspecies) )
strf(:,:) = cmplx(0.d0,0.d0,kind=8)
DO ia = 1,Na
  isp = atm2species(ia)
  DO ig = 1,Ng
    GX = (Xpos(1,ia)-shiftx)*Gv(1,ig) + (Xpos(2,ia)-shifty)*Gv(2,ig) + &
         (Xpos(3,ia)-shiftz)*Gv(3,ig)
    strf(ig,isp) = strf(ig,isp) + cmplx( cos(GX), -sin(GX), kind=8 )
  ENDDO 
ENDDO 
\end{fortrancode}

Local pseudopotential term is very simple because it is diagonal
in real space.
This term is represented by the global array {\tt V\_ps\_loc}
found in file {\tt m\_hamiltonian}. Despite its name, it can
however be used to represent any local external potential
such as harmonic potential.

It contribution to total energy is simply sum over grid
points:
\begin{fortrancode}
E_ps_loc = sum( Rhoe(:) * V_ps_loc(:) )*dVol
\end{fortrancode}

Its contribution to wave function gradient is simply obtained by
point-wise multiplication with wave function expansion
coefficients.


