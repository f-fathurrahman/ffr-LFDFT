\section{Usage}

\ffrmain accepts input file in plain text format.
The structure of the input files are very similar to PWSCF
input file with minor differences.

As an example, the following input file is for LiH molecule:
\begin{minted}{text}
&CONTROL
  pseudo_dir = '../../HGH'
  etot_conv_thr = 1.0d-6
/
&SYSTEM
  ibrav = 8
  nat = 2
  ntyp = 2
  A = 8.4668d0
  B = 8.4668d0
  C = 8.4668d0
  nr1 = 45
  nr2 = 45
  nr3 = 45
/
&ELECTRONS
  KS_Solve = 'Emin_pcg'
  cg_beta = 'DY'
  electron_maxstep = 150
  mixing_beta = 0.1
  diagonalization = 'LOBPCG'
/
ATOMIC_SPECIES
Li   3.0  Li_sc.hgh
H    1.0  H.hgh
ATOMIC_POSITIONS angstrom
Li   0.0  0.0  0.0
H    1.0  0.0  0.0
\end{minted}

Other examples can be found under directory {\tt works}.

The following input variables are special to \ffrLFDFT
and not found in PWSCF. These variables are defined in
the namelist {\tt ELECTRONS}.

\begin{itemize}
\item {\tt KS\_Solve}: method to solve Kohn-Sham equation, accepted values:
\begin{itemize}
\item {\tt 'SCF'}: using diagonalization-based self-consistent iterations
\item {\tt 'Emin-pcg'}: using direct minimization based on nonlinear
conjugate gradient algorithm
\end{itemize}

\item {\tt cg\_beta}: method to calculate parameter $\beta$ in nonlinear CG minimization
used in direct Kohn-Sham energy minimization.
\begin{itemize}
\item {\tt 'FR'}: using Fletcher-Reeves formula
\item {\tt 'PR'}: using Polak-Ribiere formula
\item {\tt 'HS'}: using Hestenes-Stiefel formula
\item {\tt 'DY'}: using Dai-Yuan formula
\end{itemize}

\end{itemize}


