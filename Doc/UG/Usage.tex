\section{Usage}

\ffrLFDFT main executable, \ffrmain supports a subset of
PWSCF input file.

The following input file is for LiH molecule:
\begin{minted}{text}
&CONTROL
  pseudo_dir = '../../HGH'
  etot_conv_thr = 1.0d-6
/
&SYSTEM
  ibrav = 8
  nat = 2
  ntyp = 2
  A = 8.4668d0
  B = 8.4668d0
  C = 8.4668d0
  nr1 = 45
  nr2 = 45
  nr3 = 45
/
&ELECTRONS
  KS_Solve = 'Emin_pcg'
  cg_beta = 'DY'
  electron_maxstep = 150
  mixing_beta = 0.1
  diagonalization = 'LOBPCG'
/
ATOMIC_SPECIES
Li   3.0  Li_sc.hgh
H    1.0  H.hgh
ATOMIC_POSITIONS angstrom
Li   0.0  0.0  0.0
H    1.0  0.0  0.0
\end{minted}

Special to \ffrLFDFT, not found in PWSCF, in {\tt ELECTRONS}
\begin{itemize}
\item {\tt KS\_Solve}: method to solve Kohn-Sham equation, accepted values:
\begin{itemize}
\item {\tt 'SCF'}: using diagonalization-based self-consistent iterations
\item {\tt 'Emin-pcg'}: using direct minimization based on nonlinear
conjugate gradient algorithm
\end{itemize}

\item {\tt cg\_beta}: method to calculate parameter $\beta$ in nonlinear CG minimization
used in direct Kohn-Sham energy minimization.
\begin{itemize}
\item {\tt 'FR'}: using Fletcher-Reeves formula
\item {\tt 'PR'}: using Polak-Ribiere formula
\item {\tt 'HS'}: using Hestenes-Stiefel formula
\item {\tt 'DY'}: using Dai-Yuan formula
\end{itemize}

\end{itemize}


