
\documentclass[a4paper,10pt,fleqn]{article}
\usepackage[a4paper]{geometry}
\geometry{verbose,tmargin=2cm,bmargin=2cm,lmargin=1.5cm,rmargin=1.5cm}

\setlength{\parskip}{\smallskipamount}
\setlength{\parindent}{0pt}

\usepackage{hyperref}
\usepackage{url}
\usepackage{fancyvrb}
\usepackage{xcolor}



\definecolor{mygray}{rgb}{0.95,0.95,0.95}
\usepackage{mdframed}
\BeforeBeginEnvironment{Verbatim}{\begin{mdframed}[backgroundcolor=mygray]}
\AfterEndEnvironment{Verbatim}{\end{mdframed}}


\makeatletter
\def\PY@reset{\let\PY@it=\relax \let\PY@bf=\relax%
    \let\PY@ul=\relax \let\PY@tc=\relax%
    \let\PY@bc=\relax \let\PY@ff=\relax}
\def\PY@tok#1{\csname PY@tok@#1\endcsname}
\def\PY@toks#1+{\ifx\relax#1\empty\else%
    \PY@tok{#1}\expandafter\PY@toks\fi}
\def\PY@do#1{\PY@bc{\PY@tc{\PY@ul{%
    \PY@it{\PY@bf{\PY@ff{#1}}}}}}}
\def\PY#1#2{\PY@reset\PY@toks#1+\relax+\PY@do{#2}}

\expandafter\def\csname PY@tok@gd\endcsname{\def\PY@tc##1{\textcolor[rgb]{0.63,0.00,0.00}{##1}}}
\expandafter\def\csname PY@tok@gu\endcsname{\let\PY@bf=\textbf\def\PY@tc##1{\textcolor[rgb]{0.50,0.00,0.50}{##1}}}
\expandafter\def\csname PY@tok@gt\endcsname{\def\PY@tc##1{\textcolor[rgb]{0.00,0.27,0.87}{##1}}}
\expandafter\def\csname PY@tok@gs\endcsname{\let\PY@bf=\textbf}
\expandafter\def\csname PY@tok@gr\endcsname{\def\PY@tc##1{\textcolor[rgb]{1.00,0.00,0.00}{##1}}}
\expandafter\def\csname PY@tok@cm\endcsname{\let\PY@it=\textit\def\PY@tc##1{\textcolor[rgb]{0.25,0.50,0.50}{##1}}}
\expandafter\def\csname PY@tok@vg\endcsname{\def\PY@tc##1{\textcolor[rgb]{0.10,0.09,0.49}{##1}}}
\expandafter\def\csname PY@tok@vi\endcsname{\def\PY@tc##1{\textcolor[rgb]{0.10,0.09,0.49}{##1}}}
\expandafter\def\csname PY@tok@mh\endcsname{\def\PY@tc##1{\textcolor[rgb]{0.40,0.40,0.40}{##1}}}
\expandafter\def\csname PY@tok@cs\endcsname{\let\PY@it=\textit\def\PY@tc##1{\textcolor[rgb]{0.25,0.50,0.50}{##1}}}
\expandafter\def\csname PY@tok@ge\endcsname{\let\PY@it=\textit}
\expandafter\def\csname PY@tok@vc\endcsname{\def\PY@tc##1{\textcolor[rgb]{0.10,0.09,0.49}{##1}}}
\expandafter\def\csname PY@tok@il\endcsname{\def\PY@tc##1{\textcolor[rgb]{0.40,0.40,0.40}{##1}}}
\expandafter\def\csname PY@tok@go\endcsname{\def\PY@tc##1{\textcolor[rgb]{0.53,0.53,0.53}{##1}}}
\expandafter\def\csname PY@tok@cp\endcsname{\def\PY@tc##1{\textcolor[rgb]{0.74,0.48,0.00}{##1}}}
\expandafter\def\csname PY@tok@gi\endcsname{\def\PY@tc##1{\textcolor[rgb]{0.00,0.63,0.00}{##1}}}
\expandafter\def\csname PY@tok@gh\endcsname{\let\PY@bf=\textbf\def\PY@tc##1{\textcolor[rgb]{0.00,0.00,0.50}{##1}}}
\expandafter\def\csname PY@tok@ni\endcsname{\let\PY@bf=\textbf\def\PY@tc##1{\textcolor[rgb]{0.60,0.60,0.60}{##1}}}
\expandafter\def\csname PY@tok@nl\endcsname{\def\PY@tc##1{\textcolor[rgb]{0.63,0.63,0.00}{##1}}}
\expandafter\def\csname PY@tok@nn\endcsname{\let\PY@bf=\textbf\def\PY@tc##1{\textcolor[rgb]{0.00,0.00,1.00}{##1}}}
\expandafter\def\csname PY@tok@no\endcsname{\def\PY@tc##1{\textcolor[rgb]{0.53,0.00,0.00}{##1}}}
\expandafter\def\csname PY@tok@na\endcsname{\def\PY@tc##1{\textcolor[rgb]{0.49,0.56,0.16}{##1}}}
\expandafter\def\csname PY@tok@nb\endcsname{\def\PY@tc##1{\textcolor[rgb]{0.00,0.50,0.00}{##1}}}
\expandafter\def\csname PY@tok@nc\endcsname{\let\PY@bf=\textbf\def\PY@tc##1{\textcolor[rgb]{0.00,0.00,1.00}{##1}}}
\expandafter\def\csname PY@tok@nd\endcsname{\def\PY@tc##1{\textcolor[rgb]{0.67,0.13,1.00}{##1}}}
\expandafter\def\csname PY@tok@ne\endcsname{\let\PY@bf=\textbf\def\PY@tc##1{\textcolor[rgb]{0.82,0.25,0.23}{##1}}}
\expandafter\def\csname PY@tok@nf\endcsname{\def\PY@tc##1{\textcolor[rgb]{0.00,0.00,1.00}{##1}}}
\expandafter\def\csname PY@tok@si\endcsname{\let\PY@bf=\textbf\def\PY@tc##1{\textcolor[rgb]{0.73,0.40,0.53}{##1}}}
\expandafter\def\csname PY@tok@s2\endcsname{\def\PY@tc##1{\textcolor[rgb]{0.73,0.13,0.13}{##1}}}
\expandafter\def\csname PY@tok@nt\endcsname{\let\PY@bf=\textbf\def\PY@tc##1{\textcolor[rgb]{0.00,0.50,0.00}{##1}}}
\expandafter\def\csname PY@tok@nv\endcsname{\def\PY@tc##1{\textcolor[rgb]{0.10,0.09,0.49}{##1}}}
\expandafter\def\csname PY@tok@s1\endcsname{\def\PY@tc##1{\textcolor[rgb]{0.73,0.13,0.13}{##1}}}
\expandafter\def\csname PY@tok@ch\endcsname{\let\PY@it=\textit\def\PY@tc##1{\textcolor[rgb]{0.25,0.50,0.50}{##1}}}
\expandafter\def\csname PY@tok@m\endcsname{\def\PY@tc##1{\textcolor[rgb]{0.40,0.40,0.40}{##1}}}
\expandafter\def\csname PY@tok@gp\endcsname{\let\PY@bf=\textbf\def\PY@tc##1{\textcolor[rgb]{0.00,0.00,0.50}{##1}}}
\expandafter\def\csname PY@tok@sh\endcsname{\def\PY@tc##1{\textcolor[rgb]{0.73,0.13,0.13}{##1}}}
\expandafter\def\csname PY@tok@ow\endcsname{\let\PY@bf=\textbf\def\PY@tc##1{\textcolor[rgb]{0.67,0.13,1.00}{##1}}}
\expandafter\def\csname PY@tok@sx\endcsname{\def\PY@tc##1{\textcolor[rgb]{0.00,0.50,0.00}{##1}}}
\expandafter\def\csname PY@tok@bp\endcsname{\def\PY@tc##1{\textcolor[rgb]{0.00,0.50,0.00}{##1}}}
\expandafter\def\csname PY@tok@c1\endcsname{\let\PY@it=\textit\def\PY@tc##1{\textcolor[rgb]{0.25,0.50,0.50}{##1}}}
\expandafter\def\csname PY@tok@o\endcsname{\def\PY@tc##1{\textcolor[rgb]{0.40,0.40,0.40}{##1}}}
\expandafter\def\csname PY@tok@kc\endcsname{\let\PY@bf=\textbf\def\PY@tc##1{\textcolor[rgb]{0.00,0.50,0.00}{##1}}}
\expandafter\def\csname PY@tok@c\endcsname{\let\PY@it=\textit\def\PY@tc##1{\textcolor[rgb]{0.25,0.50,0.50}{##1}}}
\expandafter\def\csname PY@tok@mf\endcsname{\def\PY@tc##1{\textcolor[rgb]{0.40,0.40,0.40}{##1}}}
\expandafter\def\csname PY@tok@err\endcsname{\def\PY@bc##1{\setlength{\fboxsep}{0pt}\fcolorbox[rgb]{1.00,0.00,0.00}{1,1,1}{\strut ##1}}}
\expandafter\def\csname PY@tok@mb\endcsname{\def\PY@tc##1{\textcolor[rgb]{0.40,0.40,0.40}{##1}}}
\expandafter\def\csname PY@tok@ss\endcsname{\def\PY@tc##1{\textcolor[rgb]{0.10,0.09,0.49}{##1}}}
\expandafter\def\csname PY@tok@sr\endcsname{\def\PY@tc##1{\textcolor[rgb]{0.73,0.40,0.53}{##1}}}
\expandafter\def\csname PY@tok@mo\endcsname{\def\PY@tc##1{\textcolor[rgb]{0.40,0.40,0.40}{##1}}}
\expandafter\def\csname PY@tok@kd\endcsname{\let\PY@bf=\textbf\def\PY@tc##1{\textcolor[rgb]{0.00,0.50,0.00}{##1}}}
\expandafter\def\csname PY@tok@mi\endcsname{\def\PY@tc##1{\textcolor[rgb]{0.40,0.40,0.40}{##1}}}
\expandafter\def\csname PY@tok@kn\endcsname{\let\PY@bf=\textbf\def\PY@tc##1{\textcolor[rgb]{0.00,0.50,0.00}{##1}}}
\expandafter\def\csname PY@tok@cpf\endcsname{\let\PY@it=\textit\def\PY@tc##1{\textcolor[rgb]{0.25,0.50,0.50}{##1}}}
\expandafter\def\csname PY@tok@kr\endcsname{\let\PY@bf=\textbf\def\PY@tc##1{\textcolor[rgb]{0.00,0.50,0.00}{##1}}}
\expandafter\def\csname PY@tok@s\endcsname{\def\PY@tc##1{\textcolor[rgb]{0.73,0.13,0.13}{##1}}}
\expandafter\def\csname PY@tok@kp\endcsname{\def\PY@tc##1{\textcolor[rgb]{0.00,0.50,0.00}{##1}}}
\expandafter\def\csname PY@tok@w\endcsname{\def\PY@tc##1{\textcolor[rgb]{0.73,0.73,0.73}{##1}}}
\expandafter\def\csname PY@tok@kt\endcsname{\def\PY@tc##1{\textcolor[rgb]{0.69,0.00,0.25}{##1}}}
\expandafter\def\csname PY@tok@sc\endcsname{\def\PY@tc##1{\textcolor[rgb]{0.73,0.13,0.13}{##1}}}
\expandafter\def\csname PY@tok@sb\endcsname{\def\PY@tc##1{\textcolor[rgb]{0.73,0.13,0.13}{##1}}}
\expandafter\def\csname PY@tok@k\endcsname{\let\PY@bf=\textbf\def\PY@tc##1{\textcolor[rgb]{0.00,0.50,0.00}{##1}}}
\expandafter\def\csname PY@tok@se\endcsname{\let\PY@bf=\textbf\def\PY@tc##1{\textcolor[rgb]{0.73,0.40,0.13}{##1}}}
\expandafter\def\csname PY@tok@sd\endcsname{\let\PY@it=\textit\def\PY@tc##1{\textcolor[rgb]{0.73,0.13,0.13}{##1}}}

\def\PYZbs{\char`\\}
\def\PYZus{\char`\_}
\def\PYZob{\char`\{}
\def\PYZcb{\char`\}}
\def\PYZca{\char`\^}
\def\PYZam{\char`\&}
\def\PYZlt{\char`\<}
\def\PYZgt{\char`\>}
\def\PYZsh{\char`\#}
\def\PYZpc{\char`\%}
\def\PYZdl{\char`\$}
\def\PYZhy{\char`\-}
\def\PYZsq{\char`\'}
\def\PYZdq{\char`\"}
\def\PYZti{\char`\~}
% for compatibility with earlier versions
\def\PYZat{@}
\def\PYZlb{[}
\def\PYZrb{]}
\makeatother


\begin{document}


\section{Subroutine \texttt{setup\_ffr\_LFDFT()}}

This subroutine prepares various tasks before actually solving the Kohn-Sham
equation.


\begin{Verbatim}[commandchars=\\\{\}]
\PY{k}{SUBROUTINE }\PY{n}{setup\PYZus{}ffr\PYZus{}LFDFT}\PY{p}{(}\PY{p}{)}

  \PY{k}{USE }\PY{n}{m\PYZus{}options}\PY{p}{,} \PY{k}{ONLY} \PY{p}{:} \PY{n}{FREE\PYZus{}NABLA2}\PY{p}{,} \PY{n}{I\PYZus{}POISSON\PYZus{}SOLVE}
  \PY{k}{USE }\PY{n}{m\PYZus{}input\PYZus{}vars}\PY{p}{,} \PY{k}{ONLY} \PY{p}{:} \PY{n}{assume\PYZus{}isolated}
  \PY{c}{!}
  \PY{k+kt}{INTEGER} \PY{k+kd}{::} \PY{n}{Narg}   \PY{c}{! number of argument}
  \PY{k+kt}{INTEGER} \PY{k+kd}{::} \PY{n+nb}{iargc}  \PY{c}{! needed for several compilers}
  \PY{k+kt}{CHARACTER}\PY{p}{(}\PY{l+m+mi}{64}\PY{p}{)} \PY{k+kd}{::} \PY{n}{filein}

\end{Verbatim}
We check first number of argument(s) give to the program using built-in
function \texttt{iargc()} and save the result to variable \texttt{Narg}
Currently, we only support one argument, i.e. path to input file.
The program will stop and display error message if \texttt{Narg}
is not equal to one.

\begin{Verbatim}[commandchars=\\\{\}]
  \PY{n}{Narg} \PY{o}{=} \PY{n+nb}{iargc}\PY{p}{(}\PY{p}{)}
  \PY{k}{IF}\PY{p}{(} \PY{n}{Narg} \PY{o}{/}\PY{o}{=} \PY{l+m+mi}{1} \PY{p}{)} \PY{k}{THEN}
    \PY{k}{WRITE}\PY{p}{(}\PY{o}{*}\PY{p}{,}\PY{o}{*}\PY{p}{)}
    \PY{k}{WRITE}\PY{p}{(}\PY{o}{*}\PY{p}{,}\PY{o}{*}\PY{p}{)} \PY{l+s+s1}{\PYZsq{}ERROR: exactly one argument must be given: input file path\PYZsq{}}
    \PY{k}{STOP}
  \PY{k}{ENDIF}

\end{Verbatim}
We get the actual argument using built-in subroutine \texttt{getarg()}.

\begin{Verbatim}[commandchars=\\\{\}]
  \PY{k}{CALL }\PY{n+nb}{getarg}\PY{p}{(} \PY{l+m+mi}{1}\PY{p}{,} \PY{n}{filein} \PY{p}{)}

\end{Verbatim}
We read the input file using subroutine \texttt{read\_input()}

\begin{Verbatim}[commandchars=\\\{\}]
  \PY{k}{CALL }\PY{n}{read\PYZus{}input}\PY{p}{(} \PY{n}{filein} \PY{p}{)}

\end{Verbatim}
The following subroutine will initialize global variables related to basis function
and grid points, molecular or crystalline structures, and pseudopotentials.

\begin{Verbatim}[commandchars=\\\{\}]
  \PY{k}{CALL }\PY{n}{setup\PYZus{}from\PYZus{}input}\PY{p}{(}\PY{p}{)}

\end{Verbatim}
Various options, such as convergence criteria, choice of algorithms, etc which are
given in the input file, will be converted to internal variables (mostly defined in
module \texttt{m\_options}).

\begin{Verbatim}[commandchars=\\\{\}]
  \PY{k}{CALL }\PY{n}{setup\PYZus{}options}\PY{p}{(}\PY{p}{)}

\end{Verbatim}
The following calls will output information about molecular or crystalline
structures, pseudopotentials, and basis function and grid points.

\begin{Verbatim}[commandchars=\\\{\}]
  \PY{k}{CALL }\PY{n}{info\PYZus{}atoms}\PY{p}{(}\PY{p}{)}
  \PY{k}{CALL }\PY{n}{info\PYZus{}PsPot}\PY{p}{(}\PY{p}{)}
  \PY{k}{CALL }\PY{n}{info\PYZus{}LF3d}\PY{p}{(}\PY{p}{)}

\end{Verbatim}
This subroutine initialize nonlocal pseudopotential projectors. This subroutine
must be called after variables from \texttt{m\_LF3d} are initialized as they are
defined on grid points.

TODO/FIXME: To be consistent, this call should be made in pseudopotential setup.
(probably via subroutine \texttt{setup\_from\_input()})

\begin{Verbatim}[commandchars=\\\{\}]
  \PY{k}{CALL }\PY{n}{init\PYZus{}betaNL}\PY{p}{(}\PY{p}{)}

\end{Verbatim}
This call will initialize electronic states and occupation numbers.

\begin{Verbatim}[commandchars=\\\{\}]
  \PY{k}{CALL }\PY{n}{init\PYZus{}states}\PY{p}{(}\PY{p}{)}

\end{Verbatim}
This call will initialize and calculate structure factor $S_{f}(\mathbf{G})$.
This is required for periodic LF.

TODO/FIXME: This call should be made in \texttt{setup\_from\_input} or anther
subroutine.

\begin{Verbatim}[commandchars=\\\{\}]
  \PY{k}{CALL }\PY{n}{init\PYZus{}strfact\PYZus{}shifted}\PY{p}{(}\PY{p}{)}

\end{Verbatim}
Ewald energy

\begin{Verbatim}[commandchars=\\\{\}]
  \PY{k}{IF}\PY{p}{(} \PY{n}{assume\PYZus{}isolated} \PY{o}{==} \PY{l+s+s1}{\PYZsq{}sinc\PYZsq{}} \PY{p}{)} \PY{k}{THEN}
    \PY{k}{CALL }\PY{n}{calc\PYZus{}E\PYZus{}NN}\PY{p}{(}\PY{p}{)}
  \PY{k}{ELSE}
    \PY{k}{CALL }\PY{n}{calc\PYZus{}Ewald\PYZus{}qe}\PY{p}{(}\PY{p}{)}
  \PY{k}{ENDIF}

\end{Verbatim}
The following call will initialize various global variables (arrays) needed to
define Hamiltonian. It is mainly for storing potential terms.

\begin{Verbatim}[commandchars=\\\{\}]
  \PY{k}{CALL }\PY{n}{alloc\PYZus{}hamiltonian}\PY{p}{(}\PY{p}{)}

\end{Verbatim}
Allocate local pseudopotential. For periodic sinc LF the potential is constructed
directly on real space grid. For periodic LF, the potential is first constructed on
reciprocal space and then transformed to real space grid via inverse FFT.

\begin{Verbatim}[commandchars=\\\{\}]
  \PY{k}{IF}\PY{p}{(} \PY{n}{assume\PYZus{}isolated} \PY{o}{==} \PY{l+s+s1}{\PYZsq{}sinc\PYZsq{}} \PY{p}{)} \PY{k}{THEN}
    \PY{k}{CALL }\PY{n}{init\PYZus{}V\PYZus{}ps\PYZus{}loc}\PY{p}{(}\PY{p}{)}
  \PY{k}{ELSE}
    \PY{k}{CALL }\PY{n}{init\PYZus{}V\PYZus{}ps\PYZus{}loc\PYZus{}G}\PY{p}{(}\PY{p}{)}
  \PY{k}{ENDIF}

  \PY{c}{! Laplacian matrix}
  \PY{k}{CALL }\PY{n}{init\PYZus{}nabla2\PYZus{}sparse}\PY{p}{(}\PY{p}{)}
  \PY{c}{! ILU0 preconditioner based on kinetic matrix}
  \PY{k}{CALL }\PY{n}{init\PYZus{}ilu0\PYZus{}prec}\PY{p}{(}\PY{p}{)}

  \PY{k}{IF}\PY{p}{(} \PY{n}{FREE\PYZus{}NABLA2} \PY{p}{)} \PY{k}{THEN}
    \PY{k}{CALL }\PY{n}{dealloc\PYZus{}nabla2\PYZus{}sparse}\PY{p}{(}\PY{p}{)}
  \PY{k}{ENDIF}

  \PY{k}{IF}\PY{p}{(} \PY{n}{I\PYZus{}POISSON\PYZus{}SOLVE} \PY{o}{==} \PY{l+m+mi}{1} \PY{p}{)} \PY{k}{THEN}
    \PY{k}{CALL }\PY{n}{init\PYZus{}Poisson\PYZus{}solve\PYZus{}ISF}\PY{p}{(}\PY{p}{)}
  \PY{k}{ELSEIF}\PY{p}{(} \PY{n}{I\PYZus{}POISSON\PYZus{}SOLVE} \PY{o}{==} \PY{l+m+mi}{2} \PY{p}{)} \PY{k}{THEN}
    \PY{k}{CALL }\PY{n}{init\PYZus{}Poisson\PYZus{}solve\PYZus{}DAGE}\PY{p}{(}\PY{p}{)}
  \PY{k}{ENDIF}
\PY{k}{END }\PY{k}{SUBROUTINE}
\end{Verbatim}

\end{document}

