\documentclass[a4paper,11pt,fleqn]{extarticle}
\usepackage[a4paper]{geometry}
\geometry{verbose,tmargin=2cm,bmargin=2cm,lmargin=2cm,rmargin=2cm}

\setlength{\parskip}{\smallskipamount}
\setlength{\parindent}{0pt}

%\usepackage{fontspec}
%\defaultfontfeatures{Ligatures=TeX}
%\setmainfont{Linux Libertine O}
%\setmonofont{Fira Mono}

\usepackage{hyperref}
\usepackage{url}
\usepackage{xcolor}

\usepackage{amsmath}
\usepackage{amssymb}
\usepackage{braket}

\usepackage{minted}
%\newminted{julia}{breaklines,fontsize=\footnotesize}
\newminted{fortran}{breaklines,fontsize=\small}

\definecolor{mintedbg}{rgb}{0.95,0.95,0.95}
\usepackage{mdframed}

%\BeforeBeginEnvironment{minted}{\begin{mdframed}[backgroundcolor=mintedbg]}
%\AfterEndEnvironment{minted}{\end{mdframed}}

\newcommand{\ffrLFDFT}{{\tt ffr-LFDFT}\,}
\newcommand{\ffrmain}{{\tt ffr\_LFDFT.x}\,}


\begin{document}

\title{User Guide for {\ttfamily ffr-LFDFT}}
\author{Fadjar Fathurrahman}
\date{}
\maketitle

\tableofcontents

\section{Introduction}

Welcome to {\tt ffr-LFDFT} documentation.

{\tt ffr-LFDFT} is a poor man's program (or collection of subroutines, as of now)
to carry out electronic structure calculations based on density functional theory
and Lagrange basis set.

How to compile

How to use

input parameters ...

subroutines ... (implementation)

Add tutorial on how to use m\_LF3d module to solve Schrodinger equation
in 1d.

In LF3d periodic, only gamma-point sampling is used.

\section{Installation}

Compiling and linking



\section{Usage}

\ffrLFDFT main executable, \ffrmain supports a subset of
PWSCF input file.

\begin{minted}{text}
&CONTROL
/

&SYSTEM
/

&ELECTRONS
/

ATOMIC_SPECIES
...

ATOMIC_POSITIONS angstrom
...
\end{minted}



\section{Kohn-Sham equation}

Within LDA, Kohn-Sham energy functional can be written as:
\begin{equation}
E_{\mathrm{LDA}}\left[\{\psi_{i}(\mathbf{r})\}\right] = 
E_{\mathrm{kin}} + E_{\mathrm{ion}} + E_{\mathrm{Ha}} + E_{\mathrm{xc}}
\end{equation}
with the following energy terms.

(1) kinetic energy:
\begin{equation}
E_{\mathrm{kin}} = -\frac{1}{2}\sum_{i} \int
\psi_{i}^{*}(\mathbf{r})\,\nabla^2\,\psi_{i}(\mathbf{r})
\,\mathrm{d}\mathbf{r}
\end{equation}

(2) ion-electron interaction energy:
\begin{equation}
E_{\mathrm{ion}} = \int V_{\mathrm{ion}}(\mathbf{r})\, \rho(\mathbf{r})\,
\mathrm{d}\mathbf{r}
\end{equation}

(3) Hartree (electrostatic) energy:
\begin{equation}
E_{\mathrm{Ha}} = \int \frac{1}{2}
\dfrac{\rho(\mathbf{r})\rho(\mathbf{r}')}
{\left|\mathbf{r} - \mathbf{r}'\right|}
\mathrm{d}\mathbf{r}\mathrm{d}\mathbf{r}'
\end{equation}

(4) Exchange-correlation energy (using LDA):
\begin{equation}
E_{\mathrm{xc}} = \int \epsilon_{\mathrm{xc}}\left[\rho(\mathbf{r})\right]
\rho(\mathbf{r})\,\mathrm{d}\mathbf{r}
\end{equation}

Central to the density functional theory is the so-called Kohn-Sham
equation.
This equation can be written as:
\begin{equation}
\left[
-\frac{1}{2}\nabla^2  + V_{\mathrm{KS}}(\mathbf{r})
\right] \psi_{i}(\mathbf{r}) =
\epsilon_{i}\psi_{i}(\mathbf{r})
\end{equation}
where $\epsilon{i}$ and $\psi_{i}(\mathbf{r})$ is known as Kohn-Sham
eigenvalues and eigenvectors (orbitals).
Quantity $V_{\mathrm{KS}}$ is called the Kohn-Sham potential, which can be
written as sum of several potentials:
\begin{equation}
V_{\mathrm{KS}}(\mathbf{r}) = V_{\mathrm{ion}}(\mathbf{r}) + V_{\mathrm{Ha}}(\mathbf{r})
+ V_{\mathrm{xc}}(\mathbf{r})
\label{eq:KS-pot}
\end{equation}

$V_{\mathrm{ion}}$ denotes attractive potential between ion (or atomic nuclei)
with electrons. This potential can be written as:
\begin{equation}
V_{\mathrm{ion}}(\mathrm{r}) =
\sum_{I}^{N_{\mathrm{atoms}}}
\frac{Z_{I}}{ \left| \mathbf{r} - \mathbf{R}_{I} \right| }
\end{equation}
This potential is Coulombic and has singularities
at the ionic centers. It is generally difficult to describe this
potential fully. It is common to replace the full Coulombic potential
with softer potential which is known as pseudopotential.
There are various types or flavors of pseudopotentials.
In the current implementation, ion-electron potential, $V_{\mathrm{ion}}$
is treated by pseudopotential. HGH-type pseudopotential is employed due to the
the availability of analytic forms both in real and reciprocal space.

$V_{\mathrm{Ha}}$ is the classical Hartree potential. It is defined as
\begin{equation}
V_{\mathrm{Ha}}(\mathbf{r}) = \int
\frac{\rho(\mathbf{r}')}
{\mathbf{r} - \mathbf{r}'}\,\mathrm{d}\mathbf{r}',
\end{equation}
where $\rho(\mathbf{r})$ denotes electronic density:
\begin{equation}
\rho(\mathbf{r}) = \sum_{i}^{N_{\mathrm{occ}}}
\psi^{*}_{i}(\mathbf{r}) \psi_{i}(\mathbf{r})
\end{equation}
Alternatively, Hartree potential can also be obtained via solving Poisson equation:
\begin{equation}
\nabla^{2} V_{\mathrm{Ha}}(\mathbf{r}) = -4\pi \rho(\mathbf{r})
\end{equation}

The last term in Equation \eqref{eq:KS-pot} is exchange-correlation potential.


\section{Implementation}

In this section, we wil describe our implementation of various terms 
in Kohn-Sham equations using Lagrange basis functions.
The computer program which contains our implementation can be found in public
repository: \url{https://github.com/f-fathurrahman/ffr-LFDFT}.

\subsection{Kohn-Sham equations in Lagrange basis functions representation}

Using Lagrange basis function \ref{eq:LF_p_1d} and its extension in 3d, Kohn-Sham orbitals
at point $\mathbf{r} = (x,y,z)$ can be written as
\begin{equation}
\psi_{i_{st}}(x,y,z) = \sum_{\alpha}^{N_x} \sum_{\beta}^{N_y} \sum_{\gamma}^{N_z}
C_{\alpha\beta\gamma}^{i_{st}} L_{\alpha}(x) L_{\beta}(y) L_{\gamma}(z)
\end{equation}
%
Using this expansion, kinetic operator can be written as
\begin{align}
T_{\alpha\beta\gamma}^{\alpha'\beta'\gamma'} & = -\frac{1}{2} \sum_{i_{st}} f_{i_{st}}
\Braket{ \psi_{i_{st}} | \nabla^2 | \psi_{i} } \\
& =
-\frac{1}{2}
\sum_{i_{st}} f_{i_{st}} \sum_{\alpha\alpha'} \sum_{\beta\beta'} \sum_{\gamma\gamma'}
C^{i_{st}}_{\alpha\beta\gamma} \mathbb{L}_{\alpha\beta\gamma}^{\alpha'\beta'\gamma'}
C^{i_{st}}_{\alpha'\beta'\gamma'}
\end{align}
%
were the Laplacian matrix $\mathbb{L}_{\alpha\beta\gamma}^{\alpha'\beta'\gamma'}$
has the following form:
\begin{equation}
\mathbb{L}_{\alpha\beta\gamma}^{\alpha'\beta'\gamma'} =
D^{(2)}_{\alpha\alpha'}\delta_{\beta\beta'}\delta_{\gamma\gamma'} +
D^{(2)}_{\beta\beta'}\delta_{\alpha\alpha'}\delta_{\gamma\gamma'} +
D^{(2)}_{\gamma\gamma'}\delta_{\alpha\alpha'}\delta_{\beta\beta'}
\end{equation}
%
Specifically, for periodic Lagrange basis function $D^{(2)}_{ij}$, $i, j = \alpha, \beta, \gamma$
can be written as follows.
\begin{equation}
D^{(2)}_{ij} = -\left( \frac{2\pi}{L} \right)^2 \frac{N'}{3} \left( N' + 1 \right) \delta_{ij} \\
+ \dfrac{ \left(\dfrac{2\pi}{L}\right)^2 (-1)^{i-j}\cos\left[\dfrac{\pi(i-j)}{N}\right]}
{2\sin^2\left[\dfrac{\pi(i-j)}{N}\right]}
(1-\delta_{nn'})
\label{eq:kin1d_p}
\end{equation}
where $N' = (N-1)/2$.

The matrix representation of kinetic operator is sparse.

The remaining potential terms which are local have very simple matrix form, i.e.
diagonal. The action of potential operator to Kohn-Sham orbital at point
$(r_{\alpha\beta\gamma})$ thus can be
obtained by pointwise multiplication with the potential on that point:
\begin{equation}
V_{\mathrm{KS}}(r_{\alpha\beta\gamma}) = V_{\mathrm{ion}}(r_{\alpha\beta\gamma}) +
V_{\mathrm{Ha}}(r_{\alpha\beta\gamma}) + V_{\mathrm{xc}}(r_{\alpha\beta\gamma})
\end{equation}

\subsection{Methods to solve Kohn-Sham equations}

We implement two methods to solve the Kohn-Sham equations, namely
via the self-consistent field (SCF) iterations and
direct energy minimization.

Outline of SCF iterations:
\begin{itemize}
\item Guess density $\rho(\mathbf{r})$
\item Iterate until convergence
\begin{itemize}
\item Calculate Kohn-Sham potentials $V_{\mathrm{KS}}$ and build the Kohn-Sham
Hamiltonian $H_{\mathrm{KS}}$
\item Diagonalize $H_{\mathrm{KS}}$ to obtain $\mathrm{\psi_{i_{st}}}(\mathbf{r})$
and $\epsilon_{i_{st}}$.
\item Calculate charge density and total energy. If the calculation converges
the stop the calculation, if not iterate.
\end{itemize}
\end{itemize}

Outline of direct minimization, using 
\begin{itemize}
\item Generate guess Kohn-Sham orbitals, orthonormalize if needed.
\item Calculate charge density, build Kohn-Sham potential and calculate total energy
for this
\item Iterate until convergence:
%
\begin{itemize}
%
\item Calculate Kohn-Sham electronic gradient $\mathbf{g}_{\psi}$ and the preconditioned
gradient $\mathbf{Kg}_{\psi}$ where $\mathbf{K}$ is a preconditioner.
%
\item Calculate search direction:
\begin{equation}
\beta = \dfrac{\mathbf{g}_{\psi}^{\dagger}\mathbf{Kg}_{\psi}}
{\mathbf{g}_{\psi,\mathrm{prev}}^{\dagger}\mathbf{Kg}_{\psi,\mathrm{prev}}}
\end{equation}
If $\mathbf{g}_{\psi,\mathrm{prev}}$ is not available (first iteration) then set
$\beta = 0$.
%
\item Calculate new direction:
\begin{equation}
\mathbf{d} = 
\end{equation}
\end{itemize}
%
\end{itemize}


\appendix

\input{LF_basis}
\section{HGH pseudopotential}

HGH pseudopotential has analytic forms both in real space
and reciprocal space.

Local component of pseudopotential in real space
\begin{multline}
V_{\mathrm{loc}}(\mathbf{r}) = 
-\dfrac{Z_{\mathrm{ion}}}{r}
\mathrm{erf}\left(
\dfrac{r}{\sqrt{2}r_{\mathrm{loc}}}
\right) + \\
\exp
\left[ -\frac{1}{2}
\left( \frac{r}{r_{\mathrm{loc}}}\right)^2
\right]
\times
\left[
C_{1} +
C_{2}\left( \frac{r}{r_{\mathrm{loc}}}\right)^2 +
C_{3}\left( \frac{r}{r_{\mathrm{loc}}}\right)^4 +
C_{4}\left( \frac{r}{r_{\mathrm{loc}}}\right)^6
\right]
\end{multline}
with parameters: $r_{\mathrm{loc}}$, $C_{1}$, $C_{2}$, $C_{3}$, and $C_{4}$.

Local component of local pseudopotential in $\mathbf{G}$-space:
\begin{multline}
V_{\mathrm{loc}}(\mathbf{G}) = 
-\dfrac{1}{\Omega}
\dfrac{4\pi Z_{\mathrm{ion}}}{G^2}
\exp\left[
-\dfrac{1}{2}
\left(Gr_{\mathrm{loc}}\right)^2
\right] + 
\sqrt{8\pi^3}\dfrac{r_{\mathrm{loc}}}{\Omega}
\exp\left[
-\dfrac{1}{2}
\left(Gr_{\mathrm{loc}}\right)^2
\right] \times \\
\left\{
C_{1}
+ C_{2}\left[3-\left(Gr_{\mathrm{loc}}\right)^2\right]
+ C_{3}\left[15 - 10\left(Gr_{\mathrm{loc}}\right)^2
  \left(Gr_{\mathrm{loc}}\right)^4 \right] \right. \\
\left.
+ C_{4}\left[105 - 105\left(Gr_{\mathrm{loc}}\right)^2
  + 21\left(Gr_{\mathrm{loc}}\right)^4
  - \left(Gr_{\mathrm{loc}}\right)^6\right]
\right\}
\end{multline}

Nonlocal component of pseudopotential can be written as
\begin{equation}
V_{l}(\mathbf{r},\mathbf{r}') =
\sum_{i=1}^{3} \sum_{j=1}^{3} \sum_{m=-l}^{l}
\beta_{ilm}(\mathbf{r})\,h^{l}_{ij}\,\beta^{*}_{jlm}(\mathbf{r}')
\end{equation}
with atomic-centered functions projector functions as
\begin{equation}
\beta_{ilm}(\mathbf{r}) = 
p^{l}_{i}(r) Y_{lm}(\hat{\mathbf{r}})
\end{equation}
The radial projector functions have the following form in real space
\begin{equation}
p_{i}^{l}(r) = \frac{\sqrt{2} r^{l+2(i-1)}
\exp\left( -\dfrac{r^2}{2r_{l}^2} \right) }
{r_{l}^{l+(4i-1)/2}
\sqrt{\Gamma\left(l + \dfrac{4i-1}{2}\right)}
}
\end{equation}

The radial projector functions satisfy the following normalization condition
\begin{equation}
\int_{0}^{\infty} p_{i}^{l}(r) p_{i}^{l}(r)\, r^2\,\mathrm{d}r = 1
\end{equation}

For $l = 0$, the Fourier transform of radial projector functions can be written as:
\begin{align}
p^{l=0}_{1}(G) & = \dfrac
{4\sqrt{2r_0^3}\pi^{5/4}}
{\sqrt{\Omega}\exp\left[(Gr_0)^2/2\right]}
\\
p^{l=0}_{2}(G) & = \dfrac
{\sqrt{8\dfrac{2r_0^3}{15}}\pi^{5/4}
\left( 3-(Gr_0)^2 \right)}
{\sqrt{\Omega}\exp\left[(Gr_0)^2/2\right]}
\\
p^{l=0}_{3}(G) & = \dfrac
{16\sqrt{\dfrac{2r_0^3}{105}}\pi^{5/4}
\left(15 - 10(Gr_0)^2 - (Gr_0)^4 \right)}
{3\sqrt{\Omega}\exp\left[(Gr_0)^2/2\right]}
\end{align}

For $l=1$, the Fourier transform of radial projector functions can be written as
\begin{align}
p^{l=1}_{1}(G) & = \dfrac
{8\sqrt{\dfrac{r_1^5}{3}}\pi^{5/4}G}
{\sqrt{\Omega}\exp\left[(Gr_1)^2/2\right]}
\\
p^{l=1}_{2}(G) & = \dfrac
{16\sqrt{\dfrac{r_1^5}{105}}\pi^{5/4}
\left( 5 - (Gr_1)^2 \right)G}
{\sqrt{\Omega}\exp\left[(Gr_1)^2/2\right]}
\\
p^{l=1}_{3}(G) & = \dfrac
{32\sqrt{\dfrac{r_1^5}{1155}}\pi^{5/4}
\left( 35 - 14(Gr_1)^2 + (Gr_1)^4 \right)G}
{3\sqrt{\Omega}\exp\left[(Gr_1)^2/2\right]}
\end{align}


For $l=2$, the Fourier transform of radial projector functions can be written as
\begin{align}
p^{l=2}_{1}(G) & = \dfrac
{8\sqrt{\dfrac{2r_2^7}{15}}\pi^{5/4}G^2}
{\sqrt{\Omega}\exp\left[(Gr_2)^2/2\right]}
\\
p^{l=2}_{2}(G) & = \dfrac
{16\sqrt{\dfrac{2r_2^7}{105}}\pi^{5/4}
\left( 7 - (Gr_2)^2 \right)G^2}
{3\sqrt{\Omega}\exp\left[(Gr_2)^2/2\right]}
\end{align}

For $l=3$, the Fourier transform of radial projector function can be written as
\begin{equation}
p^{l=3}_{1}(G) = \dfrac
{16\sqrt{\dfrac{2r_3^9}{105}}\pi^{5/4}G^3}
{\sqrt{\Omega}\exp\left[(Gr_3)^2/2\right]}
\end{equation}




\end{document}


