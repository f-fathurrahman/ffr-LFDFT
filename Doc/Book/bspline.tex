\chapter{Interpolation with \texttt{bspline}}

\texttt{bspline} library by Jacob Williams.

\section{Interpolation in 1D}


\section{Interpolation in 3D}

Determines the parameters of a function that interpolates
the three-dimensional gridded data

$$ [x(i),y(j),z(k),\mathrm{fcn}(i,j,k)] ~\mathrm{for}~
   i=1,..,n_x ~\mathrm{and}~ j=1,..,n_y, ~\mathrm{and}~ k=1,..,n_z $$

The interpolating function and
its derivatives may subsequently be evaluated by the function
[[db3val]].

The interpolating function is a piecewise polynomial function
represented as a tensor product of one-dimensional b-splines. the
form of this function is

$$ s(x,y,z) = \sum_{i=1}^{n_x} \sum_{j=1}^{n_y} \sum_{k=1}^{n_z}
              a_{ijk} u_i(x) v_j(y) w_k(z) $$

where the functions \(u_i\), \(v_j\), and \(w_k\) are one-dimensional b-
spline basis functions. the coefficients \(a_{ijk}\) are chosen so that:

$$ s(x(i),y(j),z(k)) = \mathrm{fcn}(i,j,k)
   ~\mathrm{for}~ i=1,..,n_x , j=1,..,n_y , k=1,..,n_z $$

Note that for fixed values of \(y\) and \(z\) \(s(x,y,z)\) is a piecewise
polynomial function of \(x\) alone, for fixed values of \(x\) and \(z\) \(s(x,y,z)\)
is a piecewise polynomial function of \(y\) alone, and for fixed
values of \(x\) and \(y\) \(s(x,y,z)\) is a function of \(z\) alone. in one
dimension a piecewise polynomial may be created by partitioning a
given interval into subintervals and defining a distinct polynomial
piece on each one. the points where adjacent subintervals meet are
called knots. each of the functions \(u_i\), \(v_j\), and \(w_k\) above is a
piecewise polynomial.

Users of [[db3ink]] choose the order (degree+1) of the polynomial
pieces used to define the piecewise polynomial in each of the \(x\), \(y\),
and \(z\) directions (`kx`, `ky`, and `kz`). users also may define their own
knot sequence in \(x\), \(y\), \(z\) separately (`tx`, `ty`, and `tz`). if `iflag=0`,
however, [[db3ink]] will choose sequences of knots that result in a
piecewise polynomial interpolant with `kx-2` continuous partial
derivatives in \(x\), `ky-2` continuous partial derivatives in \(y\), and `kz-2`
continuous partial derivatives in \(z\). (`kx` knots are taken near
each endpoint in \(x\), not-a-knot end conditions are used, and the
remaining knots are placed at data points if `kx` is even or at
midpoints between data points if `kx` is odd. the \(y\) and \(z\) directions
are treated similarly.)
After a call to [[db3ink]], all information necessary to define the
interpolating function are contained in the parameters `nx`, `ny`, `nz`,
`kx`, `ky`, `kz`, `tx`, `ty`, `tz`, and `bcoef`. these quantities should not be
altered until after the last call of the evaluation routine [[db3val]].


\begin{fortrancode}
pure subroutine db3ink(x,nx,y,ny,z,nz,fcn,kx,ky,kz,iknot,tx,ty,tz,bcoef,iflag)

  integer,intent(in) :: nx !! number of \(x\) abcissae ( $ \ge 3 $ )
  integer,intent(in) :: ny !! number of \(y\) abcissae ( $ \ge 3 $ )
  integer,intent(in) :: nz !! number of \(z\) abcissae ( $ \ge 3 $ )

  integer,intent(in) :: kx
  !! The order of spline pieces in \(x\) ( \( 2 \le k_x < n_x \) )
  !! (order = polynomial degree + 1)

  integer,intent(in) :: ky
  !! The order of spline pieces in \(y\) ( \( 2 \le k_y < n_y \) )
  !! (order = polynomial degree + 1)

  integer,intent(in) :: kz
  !! the order of spline pieces in \(z\) ( \( 2 \le k_z < n_z \) )
  !! (order = polynomial degree + 1)

  real(8),dimension(:),intent(in) :: x
  !! `(nx)` array of $x$ abcissae. must be strictly increasing.

  real(8),dimension(:),intent(in) :: y
  !! `(ny)` array of $y$ abcissae. must be strictly increasing.

  real(8),dimension(:),intent(in) :: z
  !! `(nz)` array of $z$ abcissae. must be strictly increasing.

  real(8),dimension(:,:,:),intent(in) :: fcn
  !! `(nx,ny,nz)` matrix of function values to interpolate. `fcn(i,j,k)` should
  !! contain the function value at the point (`x(i)`,`y(j)`,`z(k)`)

  integer,intent(in) :: iknot
  !! knot sequence flag:
  !!
  !! * 0 = knot sequence chosen by [[db3ink]].
  !! * 1 = knot sequence chosen by user.

  real(8),dimension(:),intent(inout) :: tx
  !! The `(nx+kx)` knots in the \(x\) direction for the spline interpolant.
  !!
  !! * If `iknot=0` these are chosen by [[db3ink]].
  !! * If `iknot=1` these are specified by the user.
  !!
  !! Must be non-decreasing.

  real(8),dimension(:),intent(inout) :: ty
  !! The `(ny+ky)` knots in the \(y\) direction for the spline interpolant.
  !!
  !! * If `iknot=0` these are chosen by [[db3ink]].
  !! * If `iknot=1` these are specified by the user.
  !!
  !! Must be non-decreasing.

  real(8),dimension(:),intent(inout) :: tz
  !! The `(nz+kz)` knots in the $z$ direction for the spline interpolant.
  !!
  !! * If `iknot=0` these are chosen by [[db3ink]].
  !! * If `iknot=1` these are specified by the user.
  !!
  !! Must be non-decreasing.

  real(8),dimension(:,:,:),intent(out) :: bcoef
  !! `(nx,ny,nz)` matrix of coefficients of the b-spline interpolant.

  integer,intent(out) :: iflag
  !! *  0 = successful execution.
  !! *  2 = `iknot` out of range.
  !! *  3 = `nx` out of range.
  !! *  4 = `kx` out of range.
  !! *  5 = `x` not strictly increasing.
  !! *  6 = `tx` not non-decreasing.
  !! *  7 = `ny` out of range.
  !! *  8 = `ky` out of range.
  !! *  9 = `y` not strictly increasing.
  !! * 10 = `ty` not non-decreasing.
  !! * 11 = `nz` out of range.
  !! * 12 = `kz` out of range.
  !! * 13 = `z` not strictly increasing.
  !! * 14 = `ty` not non-decreasing.
  !! * 700 = `size(x) ` $\ne$ `size(fcn,1)`
  !! * 701 = `size(y) ` $\ne$ `size(fcn,2)`
  !! * 702 = `size(z) ` $\ne$ `size(fcn,3)`
  !! * 706 = `size(x) ` $\ne$ `nx`
  !! * 707 = `size(y) ` $\ne$ `ny`
  !! * 708 = `size(z) ` $\ne$ `nz`
  !! * 712 = `size(tx)` $\ne$ `nx+kx`
  !! * 713 = `size(ty)` $\ne$ `ny+ky`
  !! * 714 = `size(tz)` $\ne$ `nz+kz`
  !! * 800 = `size(x) ` $\ne$ `size(bcoef,1)`
  !! * 801 = `size(y) ` $\ne$ `size(bcoef,2)`
  !! * 802 = `size(z) ` $\ne$ `size(bcoef,3)`
\end{fortrancode}
