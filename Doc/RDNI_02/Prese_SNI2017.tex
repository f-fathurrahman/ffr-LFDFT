\documentclass[8pt]{beamer}

\begin{document}

\title{Development of a Computer Program to Solve Electronic Structure using
Lagrange Basis Functions}
\author{Fadjar Fathurrahman}
\date{29 November 2017}

\frame{\titlepage}

\begin{frame}
\frametitle{Outline}

\begin{itemize}
\item Kohn-Sham equations
\item Lagrange basis functions
\end{itemize}

\end{frame}

\begin{frame}
\frametitle{Total energy functional}

Within LDA, Kohn-Sham energy functional can be written as:
\begin{equation}
E_{\mathrm{LDA}}\left[\{\psi_{i}(\mathbf{r})\}\right] =
E_{\mathrm{kin}} + E_{\mathrm{ion}} + E_{\mathrm{Ha}} + E_{\mathrm{xc}}
\end{equation}
with the following energy terms.

(1) kinetic energy:
\begin{equation}
E_{\mathrm{kin}} = -\frac{1}{2}\sum_{i_{st}}
\int f_{i_{st}}
\psi_{i_{st}}^{*}(\mathbf{r})\,\nabla^2\,\psi_{i_{st}}(\mathbf{r})
\,\mathrm{d}\mathbf{r}
\end{equation}

(2) ion-electron interaction energy:
\begin{equation}
E_{\mathrm{ion}} = \int V_{\mathrm{ion}}(\mathbf{r})\, \rho(\mathbf{r})\,
\mathrm{d}\mathbf{r}
\end{equation}

(3) Hartree (electrostatic) energy:
\begin{equation}
E_{\mathrm{Ha}} = \int \frac{1}{2}
\dfrac{\rho(\mathbf{r})\rho(\mathbf{r}')}
{\left|\mathbf{r} - \mathbf{r}'\right|}
\mathrm{d}\mathbf{r}\mathrm{d}\mathbf{r}'
\end{equation}

(4) Exchange-correlation energy (using LDA):
\begin{equation}
E_{\mathrm{xc}} = \int \epsilon_{\mathrm{xc}}\left[\rho(\mathbf{r})\right]
\rho(\mathbf{r})\,\mathrm{d}\mathbf{r}
\end{equation}

\end{frame}


\begin{frame}
\frametitle{Kohn-Sham equations}

Central to the density functional theory is the so-called Kohn-Sham
equation.
This equation can be written as:
\begin{equation}
\left[
-\frac{1}{2}\nabla^2  + V_{\mathrm{KS}}(\mathbf{r})
\right] \psi_{i_{st}}(\mathbf{r}) =
\epsilon_{i_{st}}\psi_{i_{st}}(\mathbf{r})
\end{equation}
where $\epsilon{i_{st}}$ and $\psi_{i_{st}}(\mathbf{r})$ is known as Kohn-Sham
eigenvalues and eigenvectors (orbitals).
Quantity $V_{\mathrm{KS}}$ is called the Kohn-Sham potential, which can be
written as sum of several potentials:
\begin{equation}
V_{\mathrm{KS}}(\mathbf{r}) = V_{\mathrm{ion}}(\mathbf{r}) + V_{\mathrm{Ha}}(\mathbf{r})
+ V_{\mathrm{xc}}(\mathbf{r})
\label{eq:KS-pot}
\end{equation}

\end{frame}




\begin{frame}
\frametitle{Lagrange basis functions}

For a given interval $[0,L]$, with $L>0$, the grid points $x_{i}$
appropriate for periodic Lagrange function are given by:

\begin{equation}
x_{i}=\frac{L}{2}\frac{2i-1}{N}
\end{equation}
with $i=1,\ldots,N$. Number of points $N$ should be an odd number.

The periodic cardinal functions $L_{i}^{\mathrm{per}}(x)$, defined
at grid point $i$ are given by:
\begin{equation}
L_{i}^{\mathrm{per}}(x)=\frac{1}{\sqrt{NL}}\sum_{n=1}^{N}\cos\left(\frac{\pi}{L}(2n-N-1)(x-x_{i})\right).
\end{equation}
The expansion of periodic function in terms of Lagrange functions:
\begin{equation}
f(x)=\sum_{i=1}^{N}c_{i}L_{i}^{\mathrm{per}}(x)
\end{equation}
with expansion coefficients $c_{i}=\sqrt{L/N}f(x_{i})$. When doing
variational calculation, the cofficients $c_{i}$ are the variational
parameters. The actual function values $f(x_{i}$) at grid points
$x_{i}$ is obtained by $f(x_{i})=\sqrt{N/L}c_{i}$. The prefactor
is sometimes abbreviated by $h=L/N$ and is also referred to as scaling
factor.

\end{frame}


\begin{frame}
\frametitle{Cluster LF}

For a given interval $[A,B]$, with $B>A$, the grid points $x_{i}$
appropriate for cluster Lagrange function are given by:
\[
x_{i}=A+\frac{B-A}{N+1}i
\]
where $i=1,\ldots,N$. Number of points $N$ can be either odd or
even number.

The cluster Lagrange functions $L_{i}^{\mathrm{clu}}(x)$, defined
at grid point $i$ are given by:
\begin{equation}
L_{i}^{\mathrm{clu}}(x)=\frac{2}{\sqrt{(N+1)(B-A)}}\sum_{n=1}^{N}\sin\left(k_{n}(x_{i}-A)\right)\sin\left(k_{n}(x-A)\right).
\end{equation}
where $k_{n}=\pi n/(B-A)$. The expansion of a function $f(x)$ in
terms of cluster Lagrange functions:
\begin{equation}
f(x)=\sum_{i=1}^{N}c_{i}L_{i}^{\mathrm{clu}}(x)
\end{equation}
with expansion coefficients $c_{i}=\sqrt{(B-A)/(N+1)}f(x_{i})$. When
doing variational calculation, the cofficients $c_{i}$ are the variational
parameters. The actual function values $f(x_{i}$) at grid points
$x_{i}$ is obtained by $f(x_{i})=\sqrt{(N+1)/(B-A)}c_{i}$.

Matrix elements $D_{jl}^{(2)}$ of the second derivatives for cluster
Lagrange functions are
\begin{equation}
D_{jl}^{(2)}=\begin{cases}
-\dfrac{1}{2}\left(\dfrac{\pi}{B-A}\right)^{2}\dfrac{2(N+1)^{2}+1}{3}-\dfrac{1}{\sin^{2}\left[\pi j/(N+1)\right]} & j=l\\
-\dfrac{1}{2}\left(\dfrac{\pi}{B-A}\right)^{2}(-1)^{j-l}\left[\dfrac{1}{\sin^{2}\left[\dfrac{\pi(j-l)}{2(N+1)}\right]}-\dfrac{1}{\sin^{2}\left[\dfrac{\pi(j+l)}{2(N+1)}\right]}\right] & j\neq l
\end{cases}
\end{equation}

\end{frame}


\begin{frame}
\frametitle{Expansion in Lagrange basis function}

in 1D
Schrodinger equation
blah

\end{frame}


\begin{frame}
\frametitle{Solution to Kohn-Sham equations}

Self-consistent field

Direct minimization


\end{frame}

\begin{frame}
\frametitle{Result for Gaussian potentials}

Gaussian potential

\end{frame}

\begin{frame}
\frametitle{Hydrogen atom}

Gaussian potential

\end{frame}


\end{document}

